\section*{Introduction}

\subsection*{Background and Motivation}

Cars dominate our lives here in America. They're extremely important to how we live our lives. They also result in far more deaths and injuries than could ever possibly be acceptable for our primary form of transporation in the modern age. According to the CDC, "deaths from crashes in 2022 resulted in over \$470 billion in total costs - including medical costs and cost estimates for lives lost" (\href{https://www.cdc.gov/transportation-safety/about/index.html}{Source}). On the upside, things have gotten better over time. According to data visualizations created by the Insurance Institute for Highway Safety (IIHS), the number of motor vehicle crash deaths per 100,000 people since 1975 has shown a fairly steady decline (\href{https://www.iihs.org/topics/fatality-statistics/detail/yearly-snapshot}{Source}). Things are getting better, but that doesn't mean we shouldn't keep improving. Car deaths also aren't the only measure of driving risk, injuries are also an important thing to look at. 

When it comes to this topic in Colorado, this state isn't doing too bad. We tend to rank around the middle for deaths and injuries in the country. We could always be doing better though. From 2021-2023 Colorado saw around 100,000 car accidents per year. Of those, a quarter resulted in some degree of injury and around 700 people were killed each year. (Colorado Department of Transportation)

With this project I'm taking a step back from the individual accident level and looking at counties. Are there any variables we can use that could help us understand what results in more injuries in the state? Do socioeconomic factors play into this at all? Do poorer counties see more injuries? How about counties where commute times are longer? Any additional understanding we can glean about what's putting us at risk behind the wheel could potentially save lives and prevent life changing injuries.

\subsection*{Data Tables and Sources}

For the work done here I am pulling data from 4 separate sources. The main tables I'm using come from the Colorado Department of Transportation and contain individual records of every single motor accident in the state from 2021 to 2023. To account for heavily populated counties having inflated injury values I pull in population data from the Colorado Department of Local Affairs. For additional demographic data I pull from two sources. One source contains information about the median household income for each county and the other has information about the average commute time per county. The first comes from the National Institute on Minority Health and Health Disparities and the latter from Opendatasoft. Both are simply cleaned tables that were originally queried from the Census Bureau api. I used those sources as I was struggling to get the correct information from the api myself. As for the type of project I believe this would count as an observational study as I'm looking at car accidents that have already occured. 

It is worth noting that the data I am using for average commute time is out of date. The data I'm using is from 2017 and I'm applying it to data from 2021-2023. Any attempt I made to get up to date information on this variable through the census api resulted in only one county having any information. Therefore it isn't ideal but I opted to include this information anyway despite it being slightly out of date. If this was a more formal study this wouldn't be acceptable, but I feel this should be okay for a class project. 