\section*{Conclusion}

The conclusions I can draw from this project are somewhat interesting. We see that summer is the season with the largest coefficient which is surprising at first glance. We expect it to have the most injuries per 100k residents. I assumed it would be winter which is actually the second smallest. I believe this is due to the frequency of travel in the summer and the lackthereof in the winter. As my analysis does not in any way examine the number of accidents I can draw no deeper conclusions. I believe it would be intersting to examine which season sees the highest proportion of accidents that result in injury. 

What I find interesting is a variable that I did not include in the model. I was surprised that there was seemingly no linear relationship between commute times and injuries. It's worth noting that this doesn't necessarily mean there isn't a relationship between commute time and accidents in general, but there not being one for injuries was shocking. I was prepared to advocate for systems like remote work which reduce traffic and injury risk but can draw no such conclusions here. However, my commute time data is also out of date so it's possible this has changed. As such, it's unwise to read too deeply into this variable one way or the other. 

What I will say is the main dataset I used for this project is fascinating. I had to do a lot of aggregation to make it work for this project, but I believe there are so many insights to dig into here if we go further. This dataset contains so many variables related to the specific location of the accident. Using that information we could see which streets or intersections have the highest proportions of accidents that result in injury or death. We could see how that varies month to month or over seasons and use that information to inform variable traffic rules for those specific areas. We could find problematic areas that could benefit from modernized infrastructure. Accidents will always happen, but reducing the number of them that result in injury or death is always valuable and can always be improved on. 

Though it's difficult for me to glean too much from my model specifically, I see so much potential in this dataset now. This is why it's important that this type of information is recorded so diligantly. It's not hard to imagine a world where our government doesn't maintain this data at all and we would be so much worse off without it. Infact I was unable to use two fascinating variables related to suspected alcohol and marijuana use because only one or two counties even collects that information. We should continue to fund and advocate for the collection of this data by public institutions that benefit all of us. I worry so much data collection is in the hands of private businesses now and that keeps data from truly working for the greater public. I find this extremely concerning with the upcoming administrations plans to gut and destroy many government agencies that have been collecting valuable data for decades. There's so much good this kind of data does behind the scenes that we take for granted. We need to protect it.